\chapter{Classical Hall effect}
\section{Introduction}
In 1879 Edwin Hall discovered that a transverse voltage difference builds across a current-carrying conductor in a perpendicular magnetic field. This effect is known as the Hall effect, and the transverse voltage is known as the Hall voltage.
 \begin{figure}[h]
 \centering
\includegraphics[width=5cm, height=7cm]{Classical Hall effect.pdf}
\caption{\label{Classical Hall effect}Classical Hall effect is shown by an external voltage applied on a hole-rich material in the x-direction with a uniform magnetic field along the z-axis. The figure shows the development of the Hall voltage, $V_H$, in the y-direction.}
\end{figure}   
\section{Classical resistivity in 3-D}
To calculate resistivity classically, we make use of the Drude Model.
The resulting equation of motion of an electron is:
\begin{equation}
m \frac{d \vec{v}}{d t}=-e \vec{E}-e(\vec{v} \times \vec{B})-\frac{m \vec{v}}{\tau},    
\end{equation}
where ,$\vec{v}$ is the drift velocity, $B$ is the magnetic field, $\tau$ is the mean free time, $E$ is the Electric field.
Looking at the equilibrium solution:-
\begin{equation}
  v_x+\frac{e \tau}{m} B v_y=-\frac{e \tau}{m} E_x 
\end{equation}
\begin{equation}
 v_y-\frac{e \tau}{m} B v_x=-\frac{e \tau}{m} E_y
\end{equation}  
Writing the above equations in the matrix form:-
\begin{equation}
\left(\begin{array}{cc}
1 & \frac{e \tau B}{m} \\
\frac{-e \tau B}{m} & 1
\end{array}\right)\left(\begin{array}{l}
v_x \\
v_y
\end{array}\right)
 =-\frac{e \tau}{m}\left(\begin{array}{c}
E_x \\
E_y
\end{array}\right) \\
\end{equation} 
\begin{equation}
\label{classical drift veocity}
\left(\begin{array}{cc}
1 & \frac{e \tau B}{m} \\
\frac{-e \tau B}{m} & 1
\end{array}\right) \vec{v}  =-\frac{e \tau}{m} \vec{E}
\end{equation}
The current density $\vec{J}$ is defined as:
\begin{equation}
\vec{J}=-N e \vec{v}
\end{equation}
Using the definition of $\vec{J}$ and Ohm's law we get;-
\begin{equation}
\label{Current density}
\vec{J}  =\sigma \vec{E} \\
\text { or, } \quad \vec{E}  =\rho \vec{J}
\end{equation}
Substituting Eq.~\eqref{Current density} in Eq.~\eqref{classical drift veocity} and then we get:-
\begin{equation}
\rho=\left(\begin{array}{cc}
\rho_{x x} & \rho_{x y} \\
\rho_{y x} & \rho_{y y}
\end{array}\right)=\frac{m}{N e^2 \tau}\left(\begin{array}{cc}
1 & \frac{e \tau B}{m} \\
\frac{-e \tau B}{m} & 1
\end{array}\right)
\end{equation}
Hence,
\begin{equation}
\rho_{x x}=\frac{m}{N e^2 \tau}
\end{equation} 
\begin{equation}
\rho_{x y}=\frac{B}{N e}
\end{equation}
We can look at the differences between the Classical Hall effect and the Quantum Hall effect in the following picture.
\begin{figure}[!h]
    \centering
    \includegraphics[scale = 0.25]{Classical vs Quantum Hall effect.pdf}
    \caption{Classical Hall effect showed in 3-Dimensions(3D) and Quantum Hall effect showed in 2-Dimensions}
    \label{pic6.2}
\end{figure}